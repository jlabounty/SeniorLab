\documentclass[12pt]{article}
\usepackage{fullpage}
\usepackage{multicol}
\usepackage{graphicx}
\usepackage{epsfig}
\usepackage{amsmath}
\usepackage{amsfonts}
\usepackage{amssymb}
\usepackage{float}
\usepackage{pstricks}
\usepackage{cancel}
\usepackage[nottoc,numbib]{tocbibind}
\usepackage{units}

\title{\textbf{Lab Report: Compton Effect} \\ Something something physics}
\author{Joshua LaBounty --- 109191685 \\ Thomas Krahulik --- 109074342 \\ \smallbreak PHY 445}

\begin{document}

\maketitle

\pagebreak

\tableofcontents 

\pagebreak

\section{Background}

\begin{gather}\label{eq:energy_scatter}
	E_\gamma ' = \frac{E_\gamma}{1 + \frac{E_\gamma}{m_e c^2} (1 - \cos{\theta})}
\end{gather}

\section{Experimental Setup}

\section{Data Taking}

\section{Analysis}

\subsection{Electron Mass}

\subsection{Angular Dependance of the Scattering Probability}

\subsection{Free Electrons in Cu and Al}

\begin{appendix}

\section{Appendix: Derivation of Equation \ref{eq:energy_scatter}}

This formula can be described semi-classically using only conservation of energy and conservation of momentum. We approximate this interaction as a photon impacting an electron which is currently at rest. The conservation laws require that:

\begin{gather}
	E_\gamma + E_e = E_\gamma ' + E_e '\nonumber \\
	\mathbf{p}_\gamma = \mathbf{p}_\gamma' + \mathbf{p}_e' \nonumber
\end{gather}
Initially:
\begin{gather}
	E_e = m_e c^2 \nonumber
\end{gather}
And after the collision:
\begin{gather}
	E_e' = \sqrt{(\mathbf{p}_e' c)^2 + (m_e c^2)^2} \nonumber
\end{gather}
Substituting these quantities into conservation of energy:
\begin{gather}
	E_\gamma + m_e c^2 = E_\gamma' + \sqrt{(\mathbf{p}_e' c)^2 + (m_e c^2)^2} \nonumber \\
	\downarrow \nonumber \\
	\mathbf{p}_e^{2'} c^2  = (E_\gamma - E_\gamma' + m_e c^2)^2 - m_e^2 c^4 \nonumber
\end{gather}
Now returning to conservation of momentum:
\begin{gather}
	\mathbf{p}_e' = \mathbf{p}_\gamma - \mathbf{p}_\gamma' \nonumber
\end{gather}
Since these momenta are vector quantities, and we are only interested in a scalar, we can take the dot product:
\begin{gather}
	p_e^{2'} = \mathbf{p}_e' \cdot \mathbf{p}_e' = p_\gamma^2 + p_\gamma^{2'} - 2 p_\gamma p_\gamma' \cos{\theta} \nonumber \\
	p_e^{2'} c^2 =  p_\gamma^2 c^2 + p_\gamma^{2'} c^2 - 2 c^2 p_\gamma p_\gamma' \cos{\theta} \nonumber \\
	p_e^{2'} c^2 = E_\gamma^2 + E_\gamma^{2'} - 2 E_\gamma E_\gamma' \cos{\theta} \nonumber
\end{gather}
We can now equate these two equations for $p_e^{2'} c^2$ and find:
\begin{gather}
	(E_\gamma - E_\gamma' + m_e c^2)^2 - m_e^2 c^4 = E_\gamma^2 + E_\gamma^{2'} - 2 E_\gamma E_\gamma' \cos{\theta} \nonumber
\end{gather}
We then complete the square and reduce this equation in order to find:
\begin{gather}
	2 m_e c^2 (E_\gamma - E_\gamma') = 2 E_\gamma E_\gamma' (1- \cos{\theta}) \nonumber
\end{gather}
Which we rearrange to arrive back at Equation \ref{eq:energy_scatter}:
\begin{gather}
	E_\gamma ' = \frac{E_\gamma}{1 + \frac{E_\gamma}{m_e c^2} (1 - \cos{\theta})} \nonumber
\end{gather}

\end{appendix}

\end{document}


































